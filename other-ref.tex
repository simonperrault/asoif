\bibitem{MacLean2008}
MacLean, K. E. 2008. Foundations of Transparency in Tactile Information Design. IEEE Trans on Haptics, vol. 1:2, pp. 84-95, July-December 2008.

Results from this paper:
1- abstraction should come with reduced cognitive cost

==> calm technology: thermal feedback is a bit slower, and may thus be less distracting that a strong vibration
==> distinguishability, in that case we aim for recognizability: it matters more than people are able to accurately recognize the pattern

Representational mapping: e.g. encoding direction with spatially distinct actuators (e.g. west vs. left), which improves learnability

On the other hand, list of dimensions discussed, e.g. wave, frequency, rythm do not apply due to the different nature of thermal feedback and latency. We decided to go for single values for these dimensions.
Only common dimension is location, which we leverage for some scenarios.

Most previous work do provide abstract patterns, and we follow this direction (OmniVib, Yatani, etc...)

\bibitem{Chan2008}
Chan, A., MacLean, K. E., and McGrenere, J. 2008. Designing haptic icons to support collaborative turn-taking. In Int'l J Human Computer Studies, vol. 66, pp. 333-355, January 2008. E-publication Nov 17, 2007.

Discusses duration, magnitude, frequency, delays
==> all of that is controlled in our case

Most of metaphors could be applied to some extent:
Discuss about hot vs. cold sensation, hot conveys notion of urgency, and feels less comfortable,

Note: design of multimodal haptic icons is out of scope for this research.



\bibitem{Gallace2007}
Gallace, A., Tan, H. Z., and Spence, C. 2007. The body surface as a communication system: The state of the art after 50 years. In Presence: Teleoperators and Virtual Environments, vol. 16:6, pp. 655-676, December 2007.

Says that people can identify simple "lines" (e.g. vertical, etc...) but cannot recognize complex objects. Only experts can do complex objects. This advocates for simpler patterns (with one or two actuators) to maximize recognizability.

\bibitem{Brown2005}
Brown, L. M., Brewster, S. A., and Purchase, H. C. 2005. A first investigation into the effectiveness of Tactons. In Proceedings of 1st Worldhaptics Conference (WHC '05), pp. 167-176, Pisa, Italy, March 2005.

Simple Dimensions:
Frequency
Amplitude - needs to be set, and may cause pain
Waveform - Hard to do given latency
Duration - Controlled, suggest to use shorter (simpler) patterns for frequent notifications to maximize recognizability and time

Complex dimensions:
Rythm - similar issue
Complex waveform - cannot
Spatial location - can

Note: the rest of the experiments is out of scope (rythm and stuff)
Used roughhness for priority ~ temperature change?
Actually chunking by Miller 1956 lah
rhythm for distinct applications


\bibitem{MacLean2003}
MacLean, K. E. and Enriquez, M.. 2003. Perceptual design of haptic icons. In Proceedings of EuroHaptics, pp. 351-363, Dublin, Ireland, Eurohaptics Society 2003.

Force feedback with dimensions we cannot control
This paper suggests that two dimensions seem to give the best results, so advocate to stick to two among (temperature vs. trend vs. location)


----
Refs:

\bibitem{Brown2005}
Brown, L. M., Brewster, S. A., and Purchase, H. C. 2005. A first investigation into the effectiveness of Tactons. In Proceedings of 1st Worldhaptics Conference (WHC '05), pp. 167-176, Pisa, Italy, March 2005.

\bibitem{Chan2008}
Chan, A., MacLean, K. E., and McGrenere, J. 2008. Designing haptic icons to support collaborative turn-taking. In Int'l J Human Computer Studies, vol. 66, pp. 333-355, January 2008. E-publication Nov 17, 2007.

\bibitem{Gallace2007}
Gallace, A., Tan, H. Z., and Spence, C. 2007. The body surface as a communication system: The state of the art after 50 years. In Presence: Teleoperators and Virtual Environments, vol. 16:6, pp. 655-676, December 2007.

\bibitem{MacLean2008}
MacLean, K. E. 2008. Foundations of Transparency in Tactile Information Design. IEEE Trans on Haptics, vol. 1:2, pp. 84-95, July-December 2008.

\bibitem{MacLean2003}
MacLean, K. E. and Enriquez, M.. 2003. Perceptual design of haptic icons. In Proceedings of EuroHaptics, pp. 351-363, Dublin, Ireland, Eurohaptics Society 2003.


In line with previous literature on spatiotemporal vibrotactile patterns~\cite{Alvina2015,Yatani2009}, our primary goal is to design a set of patterns that can be easily recognized and convey information quickly. To ensure better recognizability, we decided to use many dimensions with reduced set of values for each instead of fewer dimensions with larger set of values, in order to leverage chunking~\cite{Miller1956}.
A secondary goal is to potentially simple mappings between patterns and applications/commands.

The set of available dimension for vibrotactile pattern includes frequency, amplitude, waveform, duration, rhythm and spatial location~\cite{Brown2005,Chan2008,MacLean2008}. Alvina et al.~\cite{Alvina2015} show that precisely locating a vibrotactile sensation may be problematic and propose a simple binary dimension instead of spatial location which is whether two consecutive pulses happen on the same motor.
MacLean \& Enriques investigated similar dimensions for force feedback.
Thermal feedback is a bit more limited, due to its different nature: a TEC will still remain either hot or cold after being turned off, and that kind of latency makes it hard to work with \textbf{rhythm} or \textbf{waveform}.
Similarly, \textbf{amplitude}, which can be seen as the temperature change rate may induce pain, we thus chose a single value for it.

In order to design our patterns, we considered the following dimensions:
\begin{itemize}
\item Temperature Direction \{ Hot, Cold \}
\item Location \{ Top, Right, Left, Bottom \}
\item Grouping Strategy \{ Neighbors, Opposite \} (for patterns involving two ECs)
\item Temperature Change \{1 degree/sec\} (controlled)
\item Temporality \{Simultaneous\} (for patterns involving two TECs, controlled)
\end{itemize}

We did not consider more complex spatiotemporal patterns, based on recommendations from Gallace et al.~\cite{Gallace2007} who suggested that participants may recognize simple shapes such as lines (by activating two actuators similar to two points), but only experts may detect more complex shapes. This also allows us to keep the patterns short duration-wise.

The grouping strategies are tested in two different experiment, as to minimize to keep the number of varying dimensiosn close to MacLean's~\cite{MacLean2008} recommendation, and, as results will show, to maximize recognizability.

In order to investigate the mapping between patterns and commands, we ran a design workshop. MacLean~\cite{MacLean2008} presented guidelines for haptic icons. While we use different dimensions, we aim to design a set that can be easily mapped, at hopefully a low cognitive cost.
The use of a spatial dimension allows us to create representational icons. For example, leveraging the similarities between cardinal points and our 4 TECs layout allows us to create a simple pattern set for navigation (e.g. activating the Left TECs for Left turn/West, or both Top/Left for North-West). Temperature can also be mapped with emotions~\cite{35,37}, providing another option for simple mapping: a hot stimulus could be used to encode a positive meaning, or the priority of an event.

Finally, it is important to note that we did not consider multimodal icons, as investigated by previous work~\cite{Chan2008}, as we wanted to specifically focus on thermal feedback. However, we do see a strong potential for multimodal icons combining thermal feedback and light, vibration or force feedback.